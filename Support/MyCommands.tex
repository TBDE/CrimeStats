\newcommand{\equationname}[0]{Equation}


%\newcommand{\MyCaption}[3]{\caption[#1]{\textbf{#1} #2\label{#3}}}

\newcommand{\MyAlgorithmReference}[1]{~(see Algorithm \MyRef{#1})}
\newcommand{\MyListingReferenceComplete}[1]{~(see \lstlistingname~\MyRef{#1})}
\newcommand{\MyListingReferenceCompleteInline}[1]{~see \lstlistingname~\MyRef{#1}}

\newcommand{\MyCaption}[3]{\caption[#1]{#1 #2\label{#3}}}

%%\newcommand{\MyCenterOneLine}[1]{  \begin{center}    \begin{tabular}{c}      #1\\    \end{tabular}  \end{center}}
\newcommand{\MyCenterOneLine}[1]{  \begin{center}    \begin{tabular}{p{0.75 \textwidth}}   \raggedright{ #1} \\    \end{tabular}  \end{center}}

\newcommand{\MyAppendixReference}[1]{~(see \appendixname~\MyRef{#1})}
\newcommand{\MyChapterReference}[1]{~(see \chaptername~\MyRef{#1})}
\newcommand{\MyChapterReferenceInline}[1]{ \chaptername~\MyRef{#1}~\nameref{#1}}
\newcommand{\MyEquation}[1]{\begin{equation}
\MyIncludeEquations{#1}
\end{equation}
}

\newcommand{\MyEquationInline}[1]{\MyIncludeEquations{#1}}

\newcommand{\MyInput}[1]{\textbf{\%\%\% Start of #1 \%\%\%}\input{#1}\textbf{\%\%\% End of #1 \%\%\%}}

\newcommand{\MyEquationLabeled}[2]{\begin{equation}
\MyIncludeEquations{#1}
\label{#2}
\end{equation}
}

\newcommand{\MyConstraintReference}[1]{~(see Constraint~\MyRef{#1})}

\newcommand{\MyEquationReference}[1]{~(see \equationname~\MyRef{#1})}

\newcommand{\MyEtAl}[0]{~et al.~}

\newcommand{\MyFigureReference}[1]{~(see \figurename~\MyRef{#1})}
\newcommand{\MyFigureReferenceInline}[1]{~see \figurename~\MyRef{#1}}
\newcommand{\MyFigureAndEquationReference}[2]{ (see \figurename~\MyRef{#1} and \equationname~\MyRef{#1})}

\newcommand{\MyFigureReferenceTwoOld}[2]{~(see \figurename~s~\MyRef{#1} and~\MyRef{#2})}
\newcommand{\MyFigureReferenceTwo}[2]{~(see Figures \MyRef{#1}, and~\MyRef{#2})}
\newcommand{\MyFigureReferenceTwoThrough}[2]{~(see Figures \MyRef{#1} through~\MyRef{#2})}

\newcommand{\MyFigureTableReference}[2]{~(see \figurename~\MyRef{#1} and~\tablename~\MyRef{#2})}

\newcommand{\MyFiguresOld}[4]{
\begin{figure*}
\centering
\includegraphics[width=5.00in,angle=-90]{#1}
\MyCaption{#2}{#3}{#4}
\end{figure*}
}

\newcommand{\MyFigures}[5][-90]{
\begin{figure*}
\centering
\includegraphics[width=0.5\textwidth,angle=#1]{#2}
\MyCaption{#3}{#4}{#5}
\end{figure*}
}

\newcommand{\MySectionName}[0]{Section}
\newcommand{\MyAppendixName}[0]{Appendix}


\newcommand{\MyHline}[0]{\noalign{\hrule height 2pt}}

\newcommand{\MyIncludeEquations}[1]{\ensuremath{\input{Equations/#1}}}

\newcommand{\MyIgnore}[1]{}

\newcommand{\MyLeadIn}[1]{\item \emph{#1}}

\newcommand{\MyLongtableEndhead}[0]{ \tablename\ \thetable{}. (Continued from the previous page.)}

\newcommand{\MyLongtableEndfoot}[0]{(Continued on the next page.)}

\newcommand{\MyNoteReference}[1]{~(see Note~\MyRef{#1})}

\newcommand{\MyPageReference}[1]{~(see page~\pageref{#1})}

\newcommand{\MyRelatedWork}[2]{\input RelatedWorks/#2}

\newcommand{\MySectionReference}[1]{~(see \MySectionName~\MyRef{#1})}
\newcommand{\MySectionReferenceInline}[1]{~see \MySectionName~\MyRef{#1}}

\newcommand{\MySectionNameReference}[1]{~(see \MySectionName~\MyRef{#1}, ``\nameref{#1}'', page \pageref{#1})}

\newcommand{\MySubSection}[1]{\subsection{#1}\Plots{#1}}

\newcommand{\MySubSubSection}[1]{\subsubsection{#1}\Plots{#1}}

\newcommand{\MySubsubsection}[1]{\paragraph{#1}\vspace{10pt}}

\newcommand{\MySubsubsubsection}[1]{\subparagraph{#1}}

\newcommand{\MyTableReference}[1]{~(see~\tablename~\MyRef{#1})}
\newcommand{\MyTableReferenceInline}[2][see]{~#1~\tablename~\MyRef{#2}}

\newcommand{\MyTableReferenceTwo}[2]{~(see~\tablename~\MyRef{#1}~and~\tablename~\MyRef{#2})}

\newcommand{\MySmallWorldLC}[0]{small-world}
\newcommand{\MySmallWorldIC}[0]{Small-world}
\newcommand{\MySmallWorldC}[0]{Small-World}

\newcommand{\MyUSWFullLC}[0]{unsupervised~\MySmallWorldLC }

\newcommand{\MyUSWFullCap}[0]{Unsupervised~\MySmallWorldC }

\newcommand{\MyUSWFullDesc}[0]{\MyUSWFullCap~(USW)}
\newcommand{\MyUSWFullDescLC}[0]{\MyUSWFullLC~(USW)}

\newcommand{\MyUSWAllCaps}[0]{UNSUPERVISED SMALL-WORLD}

\makeatletter
\newcommand*{\MyCompress}{\@minipagetrue}
\makeatother

 %%%%%%%%%%%   Start of limited usage commands. %%%%%%%%%%

\newcommand{\Caption}[1]{The first level graph presented to the #1 attack profile.}

\newcommand{\Plots}[2]{
\begin{figure*}
\centering
\includegraphics[width=5.0in,angle=-90]{Images/Examples/#1/Rplot001.ps}
\MyCaption{\Caption{#1}}{#2}{fig:#1}
\end{figure*}
}


\newcommand{\Elements}[1]{\cellcolor[gray]{0.5} #1}

\newcommand{\Turns}[1]{\cellcolor[gray]{0.8} #1}

\newcommand{\MyWattsStrogatz}[0]{Watts -- Strogatz}

\newcommand{\MyPageSetup}[0]{
\RequirePackage{calc}

%%%%%% define new lengths %%%%%%%
\newlength{\myrightmargin}
\newlength{\myleftmargin}
\newlength{\mytopmargin}
\newlength{\mybottommargin}

%%%%%%%%%%%%%%%%% set margins here %%%%%%%%%%%%%%%%%%%
\setlength{\myrightmargin}{1.0in}
\setlength{\myleftmargin}{1.0in}
\setlength{\mytopmargin}{1.0in}    
\setlength{\mybottommargin}{1.0in}
%% \setlength{\mybottommargin}{0.0in}
\setlength{\oddsidemargin}{0.0in}   % extra room on inside side

%%%%%%%%%%%%% calculate width variables %%%%%%%%%%%%%%
\setlength{\evensidemargin}{0 in}
\setlength{\marginparsep}{0 in}
\setlength{\marginparwidth}{0 in}
\setlength{\hoffset}{\myleftmargin - 1.0in}
\setlength{\textwidth}{8.5in -\myleftmargin -\myrightmargin -\oddsidemargin}

%%%%%%%%%%%% calculate height variables  %%%%%%%%%%%%%
\setlength{\voffset}{\mytopmargin -1.0in}
\setlength{\topmargin}{0 in}
\setlength{\headheight}{12pt}
\setlength{\headsep}{20pt}
%% \setlength{\footskip}{36pt}
\setlength{\footskip}{30pt}
\setlength{\textheight}  {11.0in-\mytopmargin-\mybottommargin-\headheight-\headsep-\footskip} 
}

%%%%%%%%%%%%% End of the limited usage commands. %%%%%%%%%%

\newcommand{\MyFloats}[0] {
%  These commands and ideas come from: http://mintaka.sdsu.edu/GF/bibliog/latex/floats.html
% Alter some LaTeX defaults for better treatment of figures:
    % See p.105 of "TeX Unbound" for suggested values.
    % See pp. 199-200 of Lamport's "LaTeX" book for details.
    %   General parameters, for ALL pages:
    \renewcommand{\topfraction}{0.9}	% max fraction of floats at top
    \renewcommand{\bottomfraction}{0.8}	% max fraction of floats at bottom
    %   Parameters for TEXT pages (not float pages):
    \setcounter{topnumber}{2}
    \setcounter{bottomnumber}{2}
    \setcounter{totalnumber}{4}     % 2 may work better
    \setcounter{dbltopnumber}{2}    % for 2-column pages
    \renewcommand{\dbltopfraction}{0.9}	% fit big float above 2-col. text
%    \renewcommand{\textfraction}{0.07}	% allow minimal text w. figs
    \renewcommand{\textfraction}{0.01}	% allow minimal text w. figs
    %   Parameters for FLOAT pages (not text pages):
    \renewcommand{\floatpagefraction}{0.7}	% require fuller float pages
	% N.B.: floatpagefraction MUST be less than topfraction !!
    \renewcommand{\dblfloatpagefraction}{0.7}	% require fuller float pages

	% remember to use [htp] or [htpb] for placement
}

\newcommand{\MyTopRule}[0]{\toprule[2 pt]}
\newcommand{\MyBottomRule}[0]{\toprule[2 pt]}

\newcommand{\MyNeedsReview}[1]{\huge \textbf{Needs review (\#\arabic{NeedsReviewCounter})}. #1 \normalsize\stepcounter{NeedsReviewCounter}}
\newcounter{NeedsReviewCounter}
\setcounter{NeedsReviewCounter}{1}

\newcommand{\MyLongtableSetup}[7][(Last page.)]{\MyLongtableWorker{#2}{#3}{#4}{#5}{#6}{#7}{#1}
}

\newcommand{\MyLongtableWorker}[7]{
  \begin{center}
    \setlength{\extrarowheight}{3pt}
    \begin{longtable}{#6}
      \MyCaption{#3}{#4}{#5} \\
      \MyTopRule
        #1 \\
      \hline
      \endfirsthead

      \multicolumn{#2}{c}{\MyLongtableEndhead} \\
      \MyTopRule
      #1 \\
      \hline
      \endhead

      \hline 
      \multicolumn{#2}{r}{\MyLongtableEndfoot} \\
      \endfoot

      \MyBottomRule
      \multicolumn{#2}{r}{#7} \\
      \endlastfoot
}

\newcommand{\MyLongtableSetupOld}[6]{  %%[(Last page.)]{
  \begin{center}
    \setlength{\extrarowheight}{10pt}
    \begin{longtable}{#6}
      \MyCaption{#3}{#4}{#5} \\
      \MyTopRule
        #1 \\
      \hline
      \endfirsthead

      \multicolumn{#2}{c}{\MyLongtableEndhead} \\
      \MyTopRule
      #1 \\
      \hline
      \endhead

      \hline 
      \multicolumn{#2}{r}{\MyLongtableEndfoot} \\
      \endfoot

      \MyBottomRule
      \multicolumn{#2}{r}{(Last page.)} \\
      \endlastfoot
}


\newcommand{\MyLongtableWrapup}{
    \end{longtable}
    \end{center}
}

\RequirePackage{array}
\newcolumntype{L}[1]{>{\raggedright\let\newline\\\arraybackslash\hspace{0pt}}m{#1}}
\newcolumntype{C}[1]{>{\centering\let\newline\\\arraybackslash\hspace{0pt}}m{#1}}
\newcolumntype{R}[1]{>{\raggedleft\let\newline\\\arraybackslash\hspace{0pt}}m{#1}}


\newcommand{\MyQuotationOld}[3]{
  \begin{quotation}
    \textit{#1} \\   
    \hfill \mbox{{\normalfont #2 \cite{#3}}}
  \end{quotation}
}

\newcommand{\MyQuotation}[3]{
  \begin{quotation}
    \textit{#1}
    \begin{flushright}
      #2 \cite{#3}
    \end{flushright}
  \end{quotation}
}

\newlength{\textundbildtextheight}
 
\newcommand{\textundbild}[2]{
\settototalheight\textundbildtextheight{\vbox{#1}}
#1
\vfill
\begin{center}
\includegraphics[width=\textwidth,keepaspectratio=true,height=\textheight-\the\textundbildtextheight]{#2}
\end{center}
\vfill
}

\def\MyChangeMargin#1#2{\list{}{\rightmargin#2\leftmargin#1}\item[]}
\let\endMyChangeMargin=\endlist

\newcommand{\MyRef}[1]{\ref{#1}}

\newcounter{ThumbIndexCounter}
\setcounter{ThumbIndexCounter}{0}
\newcommand{\MyThumbIndex}[0]{
  \stepcounter{ThumbIndexCounter}
  \label{thmb:{\theThumbIndexCounter}}
  \addthumb{\@currentlabelname}{\ref{thmb:{\theThumbIndexCounter}}}{white}{black}
}

\RequirePackage{xcolor}
\definecolor{MyBlue}{RGB}{40,96,139}
\definecolor{MyRed}{RGB}{255,0,0}

\makeatletter
\newenvironment{MyBlock}[2][\linewidth]
  {\begin{mdframed}[
  align=center,
  skipabove=\topsep,
  skipbelow=\topsep,
  roundcorner=5pt,
  shadow=true,
  shadowsize=4pt,
  frametitle=\phantom{#2},
  frametitlebelowskip=15pt,
  font=\sffamily,
  innerleftmargin=12pt,
  innerrightmargin=12pt,
  frametitlefont=\Large\sffamily\bfseries\color{white},
  singleextra={
%  \path[top color=MyBlue!30,bottom color=MyBlue!0.5,rounded corners]
  \path[top color=MyRed,bottom color=MyBlue!0.5,rounded corners]
    ([xshift=\pgflinewidth,yshift=-\pgflinewidth]O|-P) rectangle
    ([xshift=-\pgflinewidth,yshift=-20pt]P) -- cycle;
  \path[fill=MyBlue]
    ([xshift=5pt,yshift=-\pgflinewidth]O|-P.north west) to[out=0,in=180]
    ([xshift=20pt,yshift=-25pt]O|-P.north west) --
    ([xshift=-20pt,yshift=-25pt]P.south east) to[out=0,in=180]
    ([xshift=-5pt,yshift=-\pgflinewidth]P.north east) -- cycle;
  \path let \p1=(P), \p2= (O) in 
    node at ([yshift=-12pt]0.5*\x1+0.5*\x2,\y1) {\parbox{\dimexpr\mdf@userdefinedwidth@length-40pt\relax
}{\centering\mdf@frametitlefont#2}};
  },
  userdefinedwidth=#1]}
  {\end{mdframed}}
\makeatother

\newcommand{\MyUserNotification}[3] {
%\def\adjboxvtop{\ht\strutbox} 
\def\adjboxvtop{0.2\height} 
 \begin{table}[h!]
    \centering
    \begin{tabular}{cp{0.5\textwidth}}
      \MyHline
%%      \includegraphics[valign=t,scale=#2]{#1} & \textrm{#3} \\
      \includegraphics[valign=t,width=0.15\textwidth]{#1} & \textrm{#3} \\
      \MyHline
    \end{tabular}
  \end{table}
}

\newcommand{\MyBestPractice}[1] {\MyUserNotification{Images/bestPractice}{0.25}{#1}}
\newcommand{\MyNote}[1] {\MyUserNotification{Images/post_up_note}{0.25}{#1}}

\newcommand{\MyDefinedAs}[0]{\ensuremath{\stackrel{\text{def.}}{\hbox{\equalsfill}}}}

\newcommand{\MyColorBoxWorker}[3]
{
  \begin{tcolorbox}[enhanced,title={#1},
attach boxed title to top center={yshift=-\tcboxedtitleheight/2},
boxed title style={colback=#3}]
#2
\end{tcolorbox}
}

\newcommand{\MyColorBoxBlue}[2]
{
  \MyColorBoxWorker{#1}{#2}{blue}
}

\newcommand{\MyTempCommand}[0]{}

\newenvironment{tight_itemize}{
\begin{itemize}
  \setlength{\itemsep}{0pt}
  \setlength{\parskip}{0pt}
}{\end{itemize}}


\makeatletter
\def\equalsfill{$\m@th\mathord=\mkern-7mu
\cleaders\hbox{$\!\mathord=\!$}\hfill
\mkern-7mu\mathord=$}
\makeatother
