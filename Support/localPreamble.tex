\usepackage{graphicx}
\usepackage{multicol}
\usepackage{multirow}
%% \usepackage[yyyymmdd,hhmmss]{datetime}
\usepackage[us,hhmmss]{datetime}
\usepackage{longtable}
\usepackage{numprint}
\usepackage{booktabs}
\usepackage{url}
\usepackage[
        bookmarks=false,
        pdfpagelabels=true,
        pdfauthor={Chuck Cartledge, 757-633-2581},
        pdftitle={Using Crime Statistics to Recommend Police Precinct Locations},
        pdfsubject={Using Crime Statistics to Recommend Police Precinct Locations},
        pdfproducer={Latex with hyperref},
        pdfcreator={dvipdfm},
        pdfdisplaydoctitle=true,
        ]{hyperref}

\usepackage{attachfile}
\usepackage{lscape}

\usepackage{caption}
\usepackage{varioref}

\usepackage{latexsym}
%% \usepackage{nameref}
\usepackage{amssymb}
\usepackage[lined,boxed]{algorithm2e}
\usepackage{subfigure}
\usepackage{tikz}
\usetikzlibrary{shadows,calc}

\usepackage{paralist}

%% \usepackage{algorithmicx}
%% \usepackage{algorithms}

\def\shadowshift{3pt,-3pt}
\def\shadowradius{6pt}

\colorlet{innercolor}{black!60}
%% \colorlet{outercolor}{gray!05}
%% \colorlet{outercolor}{red!50}
\colorlet{outercolor}{blue!05}

\newcommand\drawshadow[1]{
    \begin{pgfonlayer}{shadow}
        \shade[outercolor,inner color=innercolor,outer color=outercolor] ($(#1.south west)+(\shadowshift)+(\shadowradius/2,\shadowradius/2)$) circle (\shadowradius);
        \shade[outercolor,inner color=innercolor,outer color=outercolor] ($(#1.north west)+(\shadowshift)+(\shadowradius/2,-\shadowradius/2)$) circle (\shadowradius);
        \shade[outercolor,inner color=innercolor,outer color=outercolor] ($(#1.south east)+(\shadowshift)+(-\shadowradius/2,\shadowradius/2)$) circle (\shadowradius);
        \shade[outercolor,inner color=innercolor,outer color=outercolor] ($(#1.north east)+(\shadowshift)+(-\shadowradius/2,-\shadowradius/2)$) circle (\shadowradius);
        \shade[top color=innercolor,bottom color=outercolor] ($(#1.south west)+(\shadowshift)+(\shadowradius/2,-\shadowradius/2)$) rectangle ($(#1.south east)+(\shadowshift)+(-\shadowradius/2,\shadowradius/2)$);
        \shade[left color=innercolor,right color=outercolor] ($(#1.south east)+(\shadowshift)+(-\shadowradius/2,\shadowradius/2)$) rectangle ($(#1.north east)+(\shadowshift)+(\shadowradius/2,-\shadowradius/2)$);
        \shade[bottom color=innercolor,top color=outercolor] ($(#1.north west)+(\shadowshift)+(\shadowradius/2,-\shadowradius/2)$) rectangle ($(#1.north east)+(\shadowshift)+(-\shadowradius/2,\shadowradius/2)$);
        \shade[outercolor,right color=innercolor,left color=outercolor] ($(#1.south west)+(\shadowshift)+(-\shadowradius/2,\shadowradius/2)$) rectangle ($(#1.north west)+(\shadowshift)+(\shadowradius/2,-\shadowradius/2)$);
        \filldraw ($(#1.south west)+(\shadowshift)+(\shadowradius/2,\shadowradius/2)$) rectangle ($(#1.north east)+(\shadowshift)-(\shadowradius/2,\shadowradius/2)$);
    \end{pgfonlayer}
}

% create a shadow layer, so that we don't need to worry about overdrawing other things
\pgfdeclarelayer{shadow} 
\pgfsetlayers{shadow,main}

\newcommand\shadowimage[2][]{%
\begin{tikzpicture}
\node[anchor=south west,inner sep=0] (image) at (0,0) {\includegraphics[#1]{#2}};
\drawshadow{image}
\end{tikzpicture}}

\npstyleenglish

%% \frenchspacing
\nonfrenchspacing

