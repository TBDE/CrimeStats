Cities and governments are responsible for the safety and security of their citizens.  They are also responsible to use their money responsibly.  From a city planner's perspective, one question that combines both of these needs is: where should new police precints be located?  This simply stated problem is complicated because some police precincts are already built and operational, so the placement of new precincts has to keep the existing ones in mind.

In this short report, we look at police incident reports and existing precinct locations for the City of Virginia Beach, VA to derive optimal locations for a small number of movable/new precincts.\footnote{The idea for this report came from converstations between the authors after a Big Data Data Analysis boot camp sponsered by Old Dominion University, Norfolk, VA.}  The basic ideas about how to compute the optimal location are applicable to a large number of areas, including: rezoning of school districts, size and shape of congressional precincts, and unconstrained placement of manufacuturing facilites.

In this investigation, we will look at the optimal placement of police precincts based on historical crime data in the city of Virginia Beach, VA.  We will present our results, our algorithm, and identify specific areas where the implementation could be improved.
